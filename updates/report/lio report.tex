\documentclass[11pt, oneside]{article}   	% use "amsart" instead of "article" for AMSLaTeX format
\usepackage{geometry}                		% See geometry.pdf to learn the layout options. There are lots.
\geometry{letterpaper}                   		% ... or a4paper or a5paper or ... 
%\geometry{landscape}                		% Activate for rotated page geometry
%\usepackage[parfill]{parskip}    		% Activate to begin paragraphs with an empty line rather than an indent
\usepackage{graphicx}				% Use pdf, png, jpg, or eps§ with pdflatex; use eps in DVI mode
								% TeX will automatically convert eps --> pdf in pdflatex		
\usepackage{amssymb}
\usepackage{amsmath}
\usepackage{mathtools} 
\usepackage{proof}
\usepackage{listings}
\usepackage{turnstile}

%SetFonts

%SetFonts


\title{Implementation Report}
\author{Wesley Nuzzo}
%\date{}							% Activate to display a given date or no date

\begin{document}
\maketitle

\section{LIO's DCLabel}

We're using the DCLabel type already defined in the LIO Library as a starting point.

A DCLabel (Disjunction Category Label) has a secrecy component and an integrity component, and would be specified like:

\begin{lstlisting}[language=Haskell]
dc = secrecy %% integrity
\end{lstlisting}

Where \texttt{secrecy} is the secrecy component and \texttt{integrity} is the integrity component.
Both components are CNFs, which express policies in Conjunctive Normal Form, e.g.:

\begin{lstlisting}[language=Haskell]
secrecy = (a1 \/ a2 \/ ...) /\ (b1 \/ b2 \/ ... ) /\ ...
\end{lstlisting}

A label $\texttt{s1 \%\% i1} \sqsubseteq \texttt{s2 \%\% i2}$ if and only if when interpreted as logical predicates, \texttt{s2} implies  \texttt{s1} and  \texttt{i1} implies \texttt{i2}.

Since we're only really interested in the secrecy component, we can assume that the integrity component is always \texttt{true}. This allows us to ignore it in our analysis.

\subsection{joins, meets and so on}

Comparing this to the normal notation for meets and joins: if $s_1$, $s_2$ and $i$ are CNFs, then:
$$(s_1\texttt{ \%\% }i) \sqcup (s_2\texttt{ \%\% }i) = (s_1\land s_2 \texttt{ \%\% } i)$$
$$(s_1\texttt{ \%\% }i) \sqcap (s_2\texttt{ \%\% }i) = (s_1\lor s_2 \texttt{ \%\% } i)$$
$$(s_1\lor s_2 \texttt{ \%\% } i) \sqsubseteq (s_k\texttt{ \%\% } i)  \sqsubseteq (s_1\land s_2 \texttt{ \%\% } i), k\in\{1,2\}$$
$$\top = (\texttt{cFalse} \texttt{ \%\% }i)$$
$$\bot = (\texttt{cTrue} \texttt{ \%\% }i)$$

\section{Normal Form for Declassification and Erasure Policies}
The idea is to extend the existing normal form to account for declassification and erasure policies. Because it's a normal form, if two policies are the same, they should have the same normal form, even if they're initially expressed differently.

Originally my idea was to do this:
$$p\nearrow^c q = (p\nearrow^c\top) \sqcap (p\sqcup q)$$
$$p\searrow^c q = p \sqcap (\top\searrow^c q)$$

However, it's not clear how to represent $\top\searrow^c\bot$ or $\bot\nearrow^c\top$ using this strategy.

So instead, I propose this:
$$p\nearrow^c q = (p\sqcup q) \sqcap (p \sqcup(\bot\nearrow^c\top))$$
$$p\searrow^c q = p \sqcap (q \sqcup (\top\searrow^c \bot))$$

Which enables us to add two terms, $\texttt{E }c$ and $\texttt{D }c$, representing $\bot\nearrow^c\top$ and $\top\searrow^c\bot$ respectively.
Every other declassification and erasure policy can be constructed from these two terms together with Principals using conjunctions and disjunctions.

\section{Proposed Changes to LIO}

First of all, it is probably necessary to implement a \texttt{Cond} type which functions similarly to an \texttt{LIORef} of \texttt{Bool}, but which doesn't allow conditions to be unset once set and which allows comparing conditions based on reference equivalence.

Given the proper implementation of \texttt{Cond}, we can implement the following type:

\begin{lstlisting}[language=Haskell]

data AtomicPolicy = L Principal | D Cond | E Cond

-- constructors

latticeFromPrincipal :: Principal -> AtomicPolicy
latticeFromPrincipal pr = L pr

latticePolicy :: String -> AtomicPolicy
latticePolicy str = latticeFromPrincipal $ principal str

-- instance
instance Eq AtomicPolicy where
	L x == L y  =  x == y
	D c == D d  =  c == d
	E c == E d  =  c == d
	x == y  =  false
\end{lstlisting}
(Code may have some errors; not sure)

If we then replace \texttt{Principal} with \texttt{AtomicCond} in the regular DCLabel code, we should have the basic support we need for label relationships.

We can also implement the following helper functions:

\begin{lstlisting}[language=Haskell]

(/>) :: (ToCNF a, ToCNF b) -> a -> b -> Cond -> CNF
(\>) :: (ToCNF a, ToCNF b) -> a -> b -> Cond -> CNF

p (/>) q c = (p /\ q) \/ (p /\ (E c))
p (\>) q c = p \/ (q /\ (D c))

\end{lstlisting}

Which would make it easier to create general declassification and erasure policies. 

% Looking through the code, 

After this, we would need to implement the \texttt{update} function, which updates a label based on the current status of its conditions. We can define this helper function:

\begin{lstlisting}[language=Haskell]
updateAtomic :: LIO AtomicPolicy -> LIO AtomicPolicy
updateAtomic (LIO (L l)) = LIO (L l)
updateAtomic (LIO (D c)) = if (readCond c) then cTrue else (D c)
updateAtomic (LIO (E c)) = if (readCond c) then cFalse else (E c)
\end{lstlisting}

Which then can be mapped to the atoms of the CNF to update the entire policy.

After this, its just a matter of calling \texttt{update} in the situations where the policy needs to be updated, as discussed in the previous report.


%\section{Obstacles and Challenges}

\end{document}